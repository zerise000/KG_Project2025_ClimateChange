\section{Entity Definition}
This phase describe the entity definition phase

\subsection{Objectives}
This is the final spet in \textit{iTelos} metodology, merging the knowledgeand data layers into a unified Knowledge Graph,
The input are: data resources, and the teleontology developed in previous phase.
To do so, we have to teke care of the heterogeneity in the data layer, in this layer heterogeneity is shown in 3 different
ways

\begin{itemize}
    \item \textbf{Entity Matching}: resolves discrepancies in the representation of real-world entities across
different datasets, considering both schema and data-level heterogeneity.
    \item \textbf{Entity Identification}: ensures each entity is uniquely identified, using either existing iden-
tifiers or constructed identifying sets.
    \item \textbf{Entity Mapping}: combines the teleontology with corresponding data values in the
datasets, generating the final KG.
\end{itemize}

\subsection{Entity Matching}
In our project, we gathered data from one single source, because it were sufficiently large to develop a KG respecting 
our purpose, this makes the entity matching problem trivial

\subsection{Entity Identification}
In this section we ensure thet each entity in the set is uniquely identified, we created specific identifiers.
In some cases we already had data property used as IDs (like weather stations), in other (like the measurements), we
created a specific ID, we want to highlight the entity of TimeInterval, which has an identifying set, composed by
staring time and ending time.

\subsection{Entity Mapping}
Entity mapping integrates the teleontology with the datasets by defining the relationships between entity types and
their data values.

We did so by \href{https://usc-isi-i2.github.io/karma/}{Karma} tool. The output of this phas is in \href{https://github.com/zerise000/KG_Project2025_ClimateChange/tree/main/Phase%204%20-%20Entity%20Definition}{Entity Definition}
of our repository, in the following images we can see a graphical representation
\begin{figure}[h!]
\begin{center}
\includegraphics[width=\textwidth]{"./Entity_1.jpeg"}
\caption{Visualization of the matching of the measurment to the teloeontology}
\label{Entity_1.jpeg}
\end{center}
\end{figure}
\newpage

\begin{figure}[h!]
\begin{center}
\includegraphics[width=\textwidth]{"./Entity_2.jpeg"}
\caption{Visualization of the matching of the weather station to the teloeontology}
\label{Entity_2.jpeg}
\end{center}
\end{figure}
\newpage