\section{Language Definition}
This section describes the language definition phase. This phase is fundamental as it ensures clarity and consistency by
addressing ambiguity and diversity in natural and domain-specific languages. In particular, it explicitly takes into
account the heterogeneity of the adopted resources, since some of the consulted datasets and references are available in
Italian, while others are provided in English.


By defining a shared domain language, this phase enables accurate data annotation, minimizes misunderstandings arising
from linguistic variations, and supports seamless integration across systems and sources. This step is therefore crucial
for effective communication and interoperability in complex, multilingual and multi-domain projects.

\subsection{concept identification}
going back to concept identification done in the purpose definition, a domain language ahs been developed. This has been
done by associating a unique identifier to each concept, and by giving a short definition of each concept for clarity.
The domain language is avaliabe in
\href{https://github.com/zerise000/KG_Project2025_ClimateChange/tree/main/Phase%202%20-%20Language%20Definition}{LenguageFormalization.xlsx}.

In this activity we focused on aligning the developed concept to already modeled concepts present on \href{https://www.wikidata.org/wiki/Wikidata:Main_Page}{Wikidata},
to ground them on external semantic resources.

Some concept did not match with preexisting ones, so we created custom definitions tailored to our purpose.

The same been done with the data property and the object property.

In this phase we have taken charge of the problem of the multilingual resources, we decided to have all the information
in one language, english.

Is also at this layer that we took care of the polysemy and sinonimicity.