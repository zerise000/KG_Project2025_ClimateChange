\section{Knowledge Definition}
The knowledge definition phase of the \textit{iTelos} methodology consists on the construction of the knowledge graph
by formalizing and aligning the information gathered so far and create a unified knowledge representation via a teleontology.

We want to go step by step, starting from the output of the previous layer, we want to build a extended entity relation model,
to explicit the entity types, the object properties and the data properties as well as the hierarchy of the entity typs.
we modified the ER model taking advantage from the developed domain language
\begin{figure}[h!]
\begin{center}
\includegraphics[width=\textwidth]{"./EER.png"}
\caption{EER diagram}
\label{EER}
\end{center}
\end{figure}
\newpage
full diagram in \href{https://github.com/zerise000/KG_Project2025_ClimateChange/tree/main/Phase%203%20-%20Knowledge%20Definition}{EER.drawio}

From this we edit the ontology by using Protègè, a specific tool that through its UI allowed us to define entity types, object properties
and data properties, to build a solid ontology, and to link our work to a well established set of concepts.
The sesult of this work can be found in \href{https://github.com/zerise000/KG_Project2025_ClimateChange/tree/main/Phase%203%20-%20Knowledge%20Definition}{UNITN2026KGMM\_Knowledge\_definition.ofn}.


in Protègè we started by creating all the entity types, ang assigning to each a specific concept ID, as seen in the language
definition phase some concept was tailored made for this project, like TempMeasurement, while some other are referred to
wikidata concepts, Wikidata was adopted as reference knowledge resource for concept alignment, due to its rich coverage of
geographical entities and environmental concepts relevant to the project domain, for these concepts we populated the exactMatch
annotation with the link to the wikidata page, we modeled the hierarchy as shown in the EER model, we also modeled Climate change
indicator as contained in the climate change concept, and extreme event as a part of the union of WindMeasurement and RainMeasurement.


We did the same with data properties populating also the domain and range records, the domain is the entity type the data property
is referred to, and the range is eather Integer, Float or String.


Lastly for the object properties again we modelled all the ID, and the domain and range records, where the domain is the
subject of the relationship and the range is the object.

\begin{figure}[h!]
\begin{center}
\includegraphics{"./Protegeetypes.jpeg"}
\caption{Protegeetypes}
\label{Protegeetypes}
\end{center}
\end{figure}
Here we can see a piece of the representation, showing the Etypes hierarchy