
\section{Project Design}
\subsection{Purpose}
The first step is the definition of the project’s objective.
We start from an informal definition of our purpose, captured in the following phrase.

\begin{quote}
``This project is about building a Knowledge Graph that brings together data from different weather stations to better
understand the climate of the Trentino area during the decade 2015-2025. It helps organize and connect meteorological data in a
smart way so users can easily explore trends and patterns. Using this graph, you can ask meaningful questions about
temperature, rainfall, and changes over time, like:
\begin{itemize}
 \item what are the most impacted area by the climate change
 \item where the major weather stations are located
 \item and how the wheater and the temperature distributions evolved''
\end{itemize}
\end{quote}
We can see how our domain of interest includes meteorological observations, spatial information about weather stations,
and temporal aspects related to climate evolution over time.
\subsection{information gathering}
We gathered information from the meteotrentino website, from which we collected data from the meteorological stations on
rainfall, temperature, humidity, wind direction, wind speed, atmospheric pressure, and solar radiation, divided by year.
Some stations provided only part of this data, while others had missing years or unreliable measurements.
TODO see http://storico.meteotrentino.it/web.htm