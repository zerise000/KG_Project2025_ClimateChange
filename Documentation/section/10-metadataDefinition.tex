\section{Metadata Definition}

Throughout the development of the project
there have been defined metadata, which allows
the reusage of informations across other
knowledge graphs. In this section we
define four type of metadata:
\begin{itemize}
	\item Language resources metadata
	\item Knowledge resources metadata
	\item KG metadata
\end{itemize}

\subsection{Language resources metadata}
Language metadata describe what kind of
ontologies were used and classify them 
based on the role they have. The metadata
include various fields:
\begin{itemize}
	\item \textit{Version}: the version of the
	ontologies used in the project
	\item \textit{Prefix}: defines a shorthand string
	that replaces an Internationalized Resource Identifier (IRI)
	\item \textit{Annotator}: the Annotator of the ontology
	\item \textit{Owner}: tells us who currently has the property
	of the generated ontology
	\item \textit{License}: the type of license applied
		for the current ontology
	\item \textit{Other Namespaces Reused}: indicates if 
		external namespaces have been reused
	\item \textit{Generation DateTime}: indicates when
		the current ontology has been generated
	\item \textit{Language}: this field 
		contains the information about which 
		natural language the ontology uses for
		the names of etypes and properties
	\item \textit{Translators}: contains all
		the translations made to other languages
	\item \textit{Keywords}: define the keywords
		used to summarize the role of the ontology
	\item \textit{DatDomain}: contains the language
		domain on which the ontology holds
	\item \textit{Validator}: tells us the 
		identity of who should review the
		work of the Annotator
	\item \textit{Reference (Tele)ontology}: defines
		the main teleontologies used.
	\item \textit{Reference UKC Version}:
	\item \textit{Project Page}: The domain type
		which this ontology belongs to
	\item \textit{Type}: the type of metadata described
\end{itemize}

The complete metadata encoded as a spreadsheet
can be found \href{ihttps://github.com/zerise000/KG_Project2025_ClimateChange/blob/main/Metadata/LanguageMetadata.xlsx}{here}

\subsection{Knowledge resources metadata}
Knowledge metadata describe the teleontology
generated during the Knowledge phase, including
some additional information that helps to frame
the resource in a defined context. The
fields of the metadata are the following:

\begin{itemize}
	\item \textit{Keyword}: a keyword that summarize 
	the purpose of the teleontology
	\item \textit{Publisher}: the publisher of the
	teleontology
	\item \textit{Domain}: indicates the type of knowledge domain where
	the teleontology can be reused
	\item \textit{Creator}: reports the complete names of the authors
	of the teleontology
	\item \textit{PublicationTimestamp}: reports the timestamp that 
	represents the exact moment when the teleontology 
	has been created
	\item \textit{Language}: describes the language on which
		the teleontology informations are encoded
	\item \textit{Version}: the version of the
		teleontology
	\item \textit{Owner}: reports the current owner of the
		teleontology. Notice that it does not
		necesseraly overlaps with the authors, because
		it is possible they give their teleontology
		to someone else
	\item \textit{Level}: describe the level of abstraction
		that the teleontology represents
	\item \textit{License}: indiecates the type of license applied
		to the teleontology
	\item \textit{Type}: indicates the kind of type of resource
		we are describing
\end{itemize}

The complete metadata encoded as a spreadsheet
can be found \href{https://github.com/zerise000/KG_Project2025_ClimateChange/blob/main/Metadata/KnowledgeMetadata.xlsx}{here}

\subsection{KG metadata}

The final KG metadata describe informations
about the project, its purpose togheter with 
various data about the authors in order to
track who did what. The structure of the metadata
will be the following:

\begin{itemize}
	\item \textit{License}: describe the license
	on which the knowledge graph is released
	\item \textit{Category}: it is the real world context
	where the Knowledge Graph can be reused
	\item \textit{Maintainer}: indicates the complete name
	of the current maintainer of the knowledge graph
	\item \textit{Author(s)}: indicates the complete name 
	of the creators of the knowledge graph
	\item \textit{Author(s) Email}: reports the email
	of the authors of the knowledge graph
\item \textit{Tags}: describes various tags that describe the knowledge
	graph on KnowDive
\item \textit{Publication Date}: indicates the
	date when the knowledge graph has been released
\end{itemize}

The complete metadata encoded as a spreadsheet
can be found \href{https://github.com/zerise000/KG_Project2025_ClimateChange/blob/main/Metadata/KGMetadata.xlsx}{here}
