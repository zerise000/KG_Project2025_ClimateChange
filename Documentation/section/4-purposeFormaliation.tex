\section{Purpose Definition}

In the following section we are taking the informal description of the purpose in the project design and use \textit{iTelos}
methodology to extract a formal description of the purpose.

\subsection{Domain}

The area of knowledge and the field of study interested are climatology and environmental data analysis, with a specific
focus on regional climate change. More precisely, the project lies at the intersection of meteorology, geospatial
analysis, and semantic data modeling through Knowledge Graphs.

\subsection{Context}

more specifically we are interested in the context of the Trentino area, during the decade 2015-2025,
analyzing local climate variations, weather patterns, and their impact on the environment and communities.

\subsection{Scenarios}
now we will continue illustrating the possible scenarios:

\begin{itemize}
  \item \textbf{Mario Rossi}  is performing a research on the effects of climate change, which could be useful for some supervisor
                              agencies specialized on the theme (e.g ESA). He needs to retrieve some data about the weather of the territory,
                              analyzing how the temperature has evolved over the years and how much it has drifted from the natural trend. 
  \item \textbf{Laura Rosa}  is planning a long journey on the mountains with her friends, and she needs to know the metheorogical data of the locations
                             they are going to visit, especially the rainfall events and the sunny days of the previous years in order to make
                             decision on the logistics
  \item \textbf{Luigi Verdi} is worried about how the climate events of the last years could affect his farming activities
                             and damage his products.Agriculture is one of the sectors
                             which has been mostly affected negatively by the climate change, so he wants to be prepared
\end{itemize}

\subsection{Personas}

Now that the scenarios have been defined it is time to see the profiles of the users mentioned above:

\begin{itemize}
  \item \textbf{Mario Rossi} is a 26 years old researcher at the Trento university, specialized in ecology and enviromental research
  \item \textbf{Laura Rosa} is a 30 years old woman who is also planning a trip on the mountains of the Trentino region, however
                            in her case the main activity is a trekking exursion with her friends
  \item \textbf{Luigi Verdi} is a 56 years old farmer, with a small field nearby Trento. His main products are
                             potatoes and grape, the latter one especially is cultivated in summer
                             and gathered during grape harvest period
\end{itemize}

\subsection{Competency Question}

Each Persona has to deal with different scenarios, which could be incapsuleted in the definition
of Competency Questions(CQ). Some of them could be formalized in the following table

\begin{center}
 \begin{tabular}{| l | p{10cm} |}
  \hline
  \textbf{Person} & \textbf{Question} \\
  \hline
   Mario Rossi & What is the trend of the temperature values in Trentino
                during the decade 2015-2025?  \\
  \hline
  Mario Rossi & How the metheorogical facilities are distributed in the territory? \\
  \hline
  Mario Rossi & Which areas in Trentino have seen the highest increase in temperature over time? \\
  \hline
  Laura Rosa & How was the temperature on the Trentino mountains
                during the winter of the last three years? \\
  \hline
  Laura Rosa &  What are the locations where rainfalls most frequently happen? \\
  \hline
  Laura Rosa &  What are the periods of the year where there are sunny days? \\
  \hline
  Luigi Verdi & Did the number of extreme climate events increased in the area around my field? \\
  \hline
  Luigi Verdi & What is the trend of humidity in the summer of the last three years? \\
  \hline
  Luigi Verdi & Did the number of rainfalls decreased over the last five years? \\
  \hline
   \end{tabular}
   \captionof{table}{Competency Questions}
\end{center}

\subsection{ER Diagram}
At this point the next step of the formalization procedure is to produce a first bone structure
which allows us to define the involved entities more clearly. Lets first define their schemas:
\begin{center}
 \begin{tabular}{| l | c |}
   \hline
   \textbf{Etype} & \textbf{Properties} \\
   \hline
   Location & \textit{\textbf{location\_ID}},\textit{gps\_coordinates},\textit{name} \\
    \hline
   Coordinates & \textit{latitude},\textit{longitude} \\
   \hline
   Weather & \textit{temperature},\textit{humidity},\textit{date},
                       \textit{wind speed},\textit{is\_sunny}  \\
   \hline
   MetheorogicalFacility & \textit{\textbf{facility\_ID}},\textit{name},\textit{gps\_coordinates},
                           \textit{inauguration\_date},\textit{still\_active} \\
   \hline
   ClimateEvent & \textit{\textbf{event\_ID}},\textit{location},\textit{weather\_status},\textit{event\_date} \\
   \hline
 \end{tabular}
 \label{Entities}
 \captionof{table}{Informal definition of the entities}
\end{center}

\begin{figure}[h!]
\begin{center}
 \includegraphics[width=0.5\textwidth]{"./ER.jpg"}
 \caption{ER diagram}
 \label{ER}
\end{center}
\end{figure}
As we can see, in the table \ref{Entities} there are some entities with a more general use, like \textit{Location} and \textit{Coordinates},
which represents respectively the place where a specific event happens and the gps coordinates of a certain
position. Next we find the weather conditions associated to a location, which includes
the temperature and the humidity, the wind speed, wheter the day was sunny or not and in the end the date
timestamp of those specific metheorogical configuration. Last we have the metheorogical facility, which
include its foundation date togheter with its position, and the dangerous climate event happened at a specific
location and it is associated with the weather conditions of that day. Notice how in this latter case we also
store a further date timestamp since we want to keep track of when the catastrophe happened, which does not
necessary match the time memorized in the Weather entity (maybe because the weather conditions are sampled
every 24 hours, and the climate event happen in the meantime). 
With the definition of the entities (altough informal) done it is possible now to define the ER diagram
that declares the relationship between the subjects involved, as in figure \ref{ER}. It is interesting
to notice how there is a one-to-one relationship between MetheorologicalFacility entity and Coordinates
while it could be sufficient to identify a facility through the location. The main reason why such decision
has been made is because it is possible that a metheorological station is not exactly placed inside a
specific location but it could be in the nearby.


\subsection{Information Gathering}
\subsection{Evaluation - Purpose Definition}
