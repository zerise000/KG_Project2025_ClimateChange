\section{Purpose Definition}

In the following section we are taking the informal description of the purpose in the project design and use \textit{iTelos}
methodology to extract a formal description of the purpose.

\subsection{Domain}

The area of knowledge and the field of study interested are climatology and environmental data analysis, with a specific
focus on regional climate change. More precisely, the project lies at the intersection of meteorology, geospatial
analysis, and semantic data modeling through Knowledge Graphs(KG).

\subsection{Context}

more specifically we are interested in the context of the Trentino area, during the decade 2015-2025,
analyzing local climate variations, weather patterns, and their impact on the environment and communities.

\subsection{Scenarios}
now we will continue illustrating the possible scenarios:

\begin{itemize}
  \item \textbf{S1: Mario Rossi}  is performing a research on the effects of climate change, which could be useful for some supervisor
                              agencies specialized on the theme (e.g ESA). He needs to retrieve some data about the weather of the territory,
                              analyzing how the temperature has evolved over the years and how much it has drifted from the natural trend.
  \item \textbf{S2: Laura Rosa}  is planning a long journey on the mountains with her friends, and she needs to know the meteorological data of the locations
                             they are going to visit, especially the rainfall events and the sunny days of the previous years in order to make
                             decision on the logistics
  \item \textbf{S3: Luigi Verdi} is worried about how the climate events of the last years could affect his farming activities
                             and damage his products.Agriculture is one of the sectors
                             which has been mostly affected negatively by the climate change, so he wants to be prepared
\end{itemize}

\subsection{Personas}

Now that the scenarios have been defined it is time to see the profiles of the users mentioned above:

\begin{itemize}
  \item \textbf{P1: Mario Rossi} is a 26 years old researcher at the Trento university, specialized in ecology and enviromental research
  \item \textbf{P2: Laura Rosa} is a 30 years old woman who is also planning a trip on the mountains of the Trentino region, however
                            in her case the main activity is a trekking exursion with her friends
  \item \textbf{P3: Luigi Verdi} is a 56 years old farmer, with a small field nearby Trento. His main products are
                             potatoes and grape, the latter one especially is cultivated in summer
                             and gathered during grape harvest period
\end{itemize}

\subsection{Competency Question}

Each Persona has to deal with different scenarios, which could be encapsulated in the definition
of Competency Questions(CQ). Some of them could be formalized in the following table

\begin{center}
 \begin{tabular}{| l | p{10cm} |}
  \hline
  \textbf{Person} & \textbf{Question} \\
  \hline
   Mario Rossi & \textbf{Q1: }What is the trend of the temperature values in Trentino
                during the decade 2015-2025?  \\
  \hline
  Mario Rossi & \textbf{Q2: }How the metheorogical facilities are distributed in the territory? \\
  \hline
  Mario Rossi & \textbf{Q3: }Which areas in Trentino have seen the highest increase in temperature over time? \\
  \hline
  Laura Rosa & \textbf{Q4: }How was the temperature on the Trentino mountains
                during the winter of the last three years? \\
  \hline
  Laura Rosa &  \textbf{Q5: }What are the locations where rainfalls most frequently happen? \\
  \hline
  Laura Rosa &  \textbf{Q6: }What are the periods of the year where there are more sun radiations? \\
  \hline
  Luigi Verdi & \textbf{Q7: }Did the number of extreme climate events increased in the area around my field? \\
  \hline
  Luigi Verdi & \textbf{Q8: }What is the trend of humidity in the summer of the last three years? \\
  \hline
  Luigi Verdi & \textbf{Q9: }Did the number of rainfalls decreased over the last five years? \\
  \hline
  Mario Rossi & \textbf{Q10: }Did the number atmospheric pressure changed over the last decade? \\
  \hline
   \end{tabular}
   \captionof{table}{Competency Questions}
\end{center}
\subsection{PurposeFormalization}

now from the scenarios, the personas, and the competency questions we extract a list of useful concepts,
During the concept identification phase, we focused on concepts that are directly required to answer the defined
competency questions and that can be grounded in the available data sources.

\begin{figure}[h!]
\centering
\includegraphics[width=\textwidth]{./PurposeFormalization.jpg}
\caption{Purpose formalization overview}
\label{fig:purpose}
\end{figure}
Full file in \href{https://github.com/zerise000/KG_Project2025_ClimateChange/tree/main/Documentation}{PurposeFormalization.csv}

Starting from the concept of \textbf{Meteorological Facility} and \textbf{Meteorological Measurement}, which are the most
prominent entities in the dataset, we modeled all the other concepts.\\

About the \textbf{Meteorological Facility} we had data about the coordinates, so we decided to model the concept of
\textbf{coordinates} and consequently the concept of \textbf{Bacino}, being the station located inside a "Bacino" in our dataset,
and the concept of \textbf{Trentino}.\\

Similarly for the \textbf{Meteorological Measurement} we modeled the concept of \textbf{Month} and \textbf{TimeInterval}, representing
the moment, with different granularity, in which the measurement took place.\\

having different types of \textbf{Meteorological Measurement}, we modeled the hierarchical concepts of \textbf{TempMeasurement},
\textbf{RainMeasurement}, \textbf{HumidityMeasurement}, \textbf{WindMeasurement}, \textbf{PressureMeasurement},
\textbf{SolarRadiationMeasurement}.\\

To ontologically split the concept of \textbf{Meteorological Facility} and \textbf{Meteorological Measurement} we modeled
the concept of \textbf{Sensor}, so the \textbf{Meteorological Facility} contains \textbf{Sensors}, which performs \textbf{Meteorological Measurement}.
We modeled the concept of \textbf{Trend}.
\textbf{Trend} is a \textbf{Climate change indicator}, which is pert of \textbf{Climate change}.\\

Some concepts refer to concepts already modeled in wikidata, therefore for this concept we linked the ID to the wikidata one,
for their nature the concept with contextual focus are not very reusable so cannot be linked to preexisting concepts

\subsection{ER Diagram}
At this point the next step of the formalization procedure is to produce a first bone structure
which allows us to define the involved entities more clearly. Lets first define their schemas:
\begin{comment}
\begin{center}
\begin{tabular}{| l | p{10cm} |}
\hline
\textbf{Etype} & \textbf{Properties} \\
\hline
MeteorologicalFacility & \textit{\textbf{Tavoletta n.}},\textit{Stazione, Coordinate Est/Nord, Latitudine, Longitudine, Note}\\
\hline
MeteorologicalMeasurement & \textit{\textbf{ID}}, \textit{Date, Air temperature, Min air temperature, Max air temperature,
Rain in ml, Tot rain in ml, Air relative humidity, Avg wind direction, Avg wind velocity, Atmospheric pressure, Tot solar radiation} \\
\hline
GeographicLocation & \textit{\textbf{ID}},\textit{nome, latitudine, longitudine}
\end{tabular}
\label{Entities}
\captionof{table}{Informal definition of the entities}
\end{center}
\end{comment}

\begin{figure}[h!]
\begin{center}
\includegraphics[width=\textwidth]{"./ER.png"}
\caption{ER diagram}
\label{ER}
\end{center}
\end{figure}

In Figure~\ref{fig:purpose}, the core entities of the Knowledge Graph represent the fundamental components required to
describe and analyze meteorological data at a regional scale.

The entity \textbf{Trentino} represents the Trentino region of Italy, while \textbf{Bacino} models the main valley
within this region. \textbf{Coordinate} represents specific points in space corresponding to the location of
meteorological facilities.

The entity \textbf{Meteorological facility} represents a weather station, which hosts one or more \textbf{Sensors},
each capable of observing specific meteorological phenomena. \textbf{Meteorological measurement} represents a generic
observation, with temporal information captured by \textbf{Month} and \textbf{TimeInterval}.
Specific types of measurements, such as temperature (average, minimum, or maximum, recorded via \textit{statisticType}),
rainfall, humidity, wind (speed and direction), pressure, and solar radiation, are modeled as specialized entities linked
to Meteorological measurement, \textbf{ExtremeEvent} represents unusually high or low values of rainfall or wind.

\textbf{Trend}: Each Trend instance represents a concrete climate indicator
computed for a specific geographic area and a specific time interval, computed from Meteorological measurements and representing
long-term variations. Each \textbf{Trend} instance is computed from more \textbf{Meteorological measurement} and is a
\textbf{ClimateChangeIndicator}, which in turn is part of \textbf{ClimateChange}, ClimateChange is modeled as a high-level
abstract concept, not directly associated with specific observations, but represented through derived indicators such as Trends.


\begin{comment}
\begin{center}
\begin{tabular}{| l | c |}
\hline
\textbf{Etype} & \textbf{Properties} \\
\hline
Location & \textit{\textbf{location\_ID}},\textit{gps\_coordinates},\textit{name} \\
\hline
Coordinates & \textit{latitude},\textit{longitude} \\
\hline
Weather & \textit{temperature},\textit{humidity},\textit{date},
\textit{wind speed},\textit{is\_sunny}  \\
\hline
MetheorogicalFacility & \textit{\textbf{facility\_ID}},\textit{name},\textit{gps\_coordinates},
\textit{inauguration\_date},\textit{still\_active} \\
\hline
ClimateEvent & \textit{\textbf{event\_ID}},\textit{location},\textit{weather\_status},\textit{event\_date} \\
\hline
\end{tabular}
\label{Entities_}
\captionof{table}{Informal definition of the entities}
\end{center}

\begin{figure}[h!]
\begin{center}
\includegraphics[width=0.5\textwidth]{"./ER.png"}
\caption{ER diagram}
\label{ER_}
\end{center}
\end{figure}
As we can see, in the table \ref{Entities} there are some entities with a more general use, like \textit{Location} and \textit{Coordinates},
which represents respectively the place where a specific event happens and the gps coordinates of a certain
position. Next we find the weather conditions associated to a location, which includes
the temperature and the humidity, the wind speed, wheter the day was sunny or not and in the end the date
timestamp of those specific metheorogical configuration. Last we have the metheorogical facility, which
include its foundation date togheter with its position, and the dangerous climate event happened at a specific
location and it is associated with the weather conditions of that day. Notice how in this latter case we also
store a further date timestamp since we want to keep track of when the catastrophe happened, which does not
necessary match the time memorized in the Weather entity (maybe because the weather conditions are sampled
every 24 hours, and the climate event happen in the meantime).
With the definition of the entities (altough informal) done it is possible now to define the ER diagram
that declares the relationship between the subjects involved, as in figure \ref{ER}. It is interesting
to notice how there is a one-to-one relationship between MetheorologicalFacility entity and Coordinates
while it could be sufficient to identify a facility through the location. The main reason why such decision
has been made is because it is possible that a metheorological station is not exactly placed inside a
specific location but it could be in the nearby.
\end{comment}

\newpage

\subsection{Information Gathering}

here we provide an overview of the data used in input for this project, including resources using different languages.

We want to gather relevant information coherent with our purpose, in order to construct a meaningful knowledge graph.

The knowledge layer of the proposed Knowledge Graph is grounded on external semantic resources. In particular,
\href{https://www.wikidata.org/wiki/Wikidata:Main_Page}{Wikidata}
is employed as the primary external knowledge source to link selected domain entities, such as Meteorological facility
(or weather station for sinonimicity), the region of Trentino and so on, to well-established real-world concepts.
Wikidata offers a collaboratively maintained, structured knowledge base that ensures semantic consistency and
interoperability with existing linked data resources.\\
These resources are not directly reused as ontologies but are referenced through semantic links to provide an external
grounding for key concepts.
The adopted formal knowledge resources can be classified as \textbf{common} resources, as Wikidata provides general-purpose
concepts widely used across multiple domains, No core or contextual domain-specific ontologies were directly reused.\\

Additional domain ontologies were considered during the design phase but were not directly adopted in order to preserve
model simplicity and avoid unnecessary complexity.\\
\\

The data layer of the Knowledge Graph is built on meteorological measurements and weather stations collected from the
open-access repository \href{http://storico.meteotrentino.it/web.htm}{meteotrentino}. The dataset provides detailed
information about weatherstations in the Trentino region, including geographic coordinates, macro-areas and station ID and name.
Measurements are composed by all of different measurement categories such as temperature, rainfall, humidity, wind,
pressure, and solar radiation, corresponding to the main meteorological phenomena relevant to climate analysis, and relative timestamp.

\subsubsection{MetereologicalFacility}
MetereologicalFacility, or weather station, represents the building located in a specific point of the Trentino region, this resource has the following attributes.
\begin{itemize}
    \item \textbf{Stazione}: (String) the name of the station
    \item \textbf{Tavoletta n.}: (Integer) the unique identifier of the station
    \item \textbf{Note}: (String) some additional information about hte station
\end{itemize}
the attributes are in italian, because the data was in italian in the website we got them.\\
The focus is \textbf{common} because it is reusable and we linked it with a Wikidata concept
\subsubsection{Coordinate}
Is the poin inside of the Trentino region in which a station is located.\\
Key attributes are
\begin{itemize}
    \item \textbf{CoordinateID}: (Integer) the unique identifier of the space point
    \item \textbf{Coordinate Est/Nord} (String )represents the distance from the north pole and from the main meridian, separated by a "/"
    \item \textbf{latitudine} (String) represents the latitude of the point
    \item \textbf{longitudine} (String) represent the longitude of the point
\end{itemize}
the attributes and the name are in italian, because the data was in italian in the website we got them.\\
The focus is \textbf{core} because it is general enough, but we didn't find a 1 to 1 correspondence
\subsubsection{Bacino}
Is a valley inside the Trentino region.\\
Key attributes
\begin{itemize}
    \item \textbf{NomeBacino}: (String) the name of the valley, also the unique identifier
\end{itemize}
the attributes and the name are in italian, because the data was in italian in the website we got them\\
The focus is \textbf{core} because it is general enough, but we didn't find a 1 to 1 correspondence
\subsubsection{Trentino}
Is the region in Italy we are focusing out project on.\\
Key attributes
\begin{itemize}
    \item \textbf{TrentinoID}: (Integer) unique identifier
\end{itemize}
The focus is \textbf{common} because it is reusable and we linked it with a Wikidata concept
\subsubsection{Sensor}
Is the hardware device used to perform the actual weather measurement, is sited inside a weather station.\\
Key attributes
\begin{itemize}
    \item \textbf{SensorID}: (integer) unique identifier
\end{itemize}
The focus is \textbf{common} because it is reusable and we linked it with a Wikidata concept
\subsubsection{MetereologicalMeasurement}
The actual measurement of a weather observable.\\
Key attributes
\begin{itemize}
    \item \textbf{MetereologicalMeasurementID}: (Integer) unique identifier
    \item \textbf{Value}: (Integer) the value of the measurement
    \item \textbf{Unit}: (String) the unit of measurement
\end{itemize}
The focus is \textbf{contextual} because is to specific to be reused, and we couldn't find any fitting preexisting concept 
\subsubsection{TempMeasurement}
A specific kind of MetereologicalMeasurement, representing the measurement of a temperature value.\\
Key attributes
\begin{itemize}
    \item \textbf{TempMeasurementID}: (Integer) unique identifier
    \item \textbf{statisticType}: (String) Determine if the value is a average, a minimum or a maximum
\end{itemize}
The focus is \textbf{contextual} because is to specific to be reused, and we couldn't find any fitting preexisting concept
\subsubsection{RainMeasurement}
A specific kind of MetereologicalMeasurement, representing the measurement of a rain value.\\
Key attributes
\begin{itemize}
    \item \textbf{RainMeasurementID}: (Integer) unique identifier
\end{itemize}
The focus is \textbf{contextual} because is to specific to be reused, and we couldn't find any fitting preexisting concept
\subsubsection{HumidityMeasurement}
A specific kind of MetereologicalMeasurement, representing the measurement of a humidity value.\\
Key attributes
\begin{itemize}
    \item \textbf{HumidityMeasurementID}: (Integer) unique identifier
\end{itemize}
The focus is \textbf{contextual} because is to specific to be reused, and we couldn't find any fitting preexisting concept
\subsubsection{WindMeasurement}
A specific kind of MetereologicalMeasurement, representing the measurement of a wind value.\\
Key attributes
\begin{itemize}
    \item \textbf{WindMeasurementID}: (Integer) unique identifier
    \item \textbf{Type}: (String) Determine if the value refers to the speed or the direction
\end{itemize}
The focus is \textbf{contextual} because is to specific to be reused, and we couldn't find any fitting preexisting concept
\subsubsection{PressureMeasurement}
A specific kind of MetereologicalMeasurement, representing the measurement of a pressure value.\\
Key attributes
\begin{itemize}
    \item \textbf{PressureMeasurementID}: (Integer) unique identifier
\end{itemize}
The focus is \textbf{contextual} because is to specific to be reused, and we couldn't find any fitting preexisting concept
\subsubsection{SolarRadiationMeasurement}
A specific kind of MetereologicalMeasurement, representing the measurement of a solar radiation value.\\
Key attributes
\begin{itemize}
    \item \textbf{SolarRadiationMeasurementID}: (Integer) unique identifier
\end{itemize}
The focus is \textbf{contextual} because is to specific to be reused, and we couldn't find any fitting preexisting concept
\subsubsection{ExtremeEvent}
A specific kind of RainMeasurement or WindMeasurement with a very high value.\\
Key attributes
\begin{itemize}
    \item \textbf{ExtremeEventID}: (Integer) unique identifier
\end{itemize}
The focus is \textbf{Common} because even if is a specific kind of a contextual Etype, this is modeled in Wikidata
\subsubsection{Month}
Irregular unit of time dividing a calendar year.\\
Key attributes
\begin{itemize}
    \item \textbf{MonthValue}: (String) The name of the month, also the unique identifier
\end{itemize}
The focus is \textbf{common} because it is reusable and we linked it with a Wikidata concept
\subsubsection{TimePeriod}
temporal extent having a beginning, an end and a duration.\\
Key attributes
\begin{itemize}
    \item \textbf{Start}: (String) The moment from which the time interval starts, part of unique identifier
    \item \textbf{End}: (String) The moment in which the time interval ends, part of unique identifier
\end{itemize}
The focus is \textbf{common} because it is reusable and we linked it with a Wikidata concept
\subsubsection{Trend}
cluster of statistical values showing how a variable is changing in time.\\
Key attributes
\begin{itemize}
    \item \textbf{TrendID}: (Integer)  unique identifier
\end{itemize}
The focus is \textbf{core} because it is reusable bat to specific and we didn't find a 1 to 1 correspondence in wikidata
\subsubsection{ClimateChangeIndicator}
A trend that shows how the climate is changing.\\
Key attributes
\begin{itemize}
    \item \textbf{ClimateChangeIndicatorID}: (Integer)  unique identifier
\end{itemize}
The focus is \textbf{Common} because it is reusable, and broadly used as a concept, we didn't find a wikidata correspondence
\subsubsection{ClimateChange}
human-caused changes to climate on Earth.\\
Key attributes
\begin{itemize}
    \item \textbf{ClimateChangeID}: (Integer)  unique identifier
\end{itemize}
The focus is \textbf{common} because it is reusable and we linked it with a Wikidata concept

\subsection{Evaluation - Purpose Definition}
We faced some issue during the purpose definition phase, mainly we had to change the modeling, because in the beginning
it was under modelled, and after this we made it more complex, reaching a model that we judged as over modelled, finding
in the end a good middle ground.\\

But still there are some peculiarity that we choose to keep for eventual future modification,
for instance the Etype PressureMeasurement was not answering any competency question, so we decided to add a specific
competency question to make it more integrated in the model, we added the competency question identified by ID Q10
