\section{Purpose Definition}

In the following section we are taking the informal description of the purpose in the project design and use \textit{iTelos}
methodology to extract a formal description of the purpose.

\subsection{Domain}

The area of knowledge and the field of study interested are climatology and environmental data analysis, with a specific
focus on regional climate change. More precisely, the project lies at the intersection of meteorology, geospatial
analysis, and semantic data modeling through Knowledge Graphs(KG).

\subsection{Context}

more specifically we are interested in the context of the Trentino area, during the decade 2015-2025,
analyzing local climate variations, weather patterns, and their impact on the environment and communities.

\subsection{Scenarios}
now we will continue illustrating the possible scenarios:

\begin{itemize}
  \item \textbf{S1: Mario Rossi}  is performing a research on the effects of climate change, which could be useful for some supervisor
                              agencies specialized on the theme (e.g ESA). He needs to retrieve some data about the weather of the territory,
                              analyzing how the temperature has evolved over the years and how much it has drifted from the natural trend. 
  \item \textbf{S2: Laura Rosa}  is planning a long journey on the mountains with her friends, and she needs to know the metheorogical data of the locations
                             they are going to visit, especially the rainfall events and the sunny days of the previous years in order to make
                             decision on the logistics
  \item \textbf{S3: Luigi Verdi} is worried about how the climate events of the last years could affect his farming activities
                             and damage his products.Agriculture is one of the sectors
                             which has been mostly affected negatively by the climate change, so he wants to be prepared
\end{itemize}

\subsection{Personas}

Now that the scenarios have been defined it is time to see the profiles of the users mentioned above:

\begin{itemize}
  \item \textbf{P1: Mario Rossi} is a 26 years old researcher at the Trento university, specialized in ecology and enviromental research
  \item \textbf{P2: Laura Rosa} is a 30 years old woman who is also planning a trip on the mountains of the Trentino region, however
                            in her case the main activity is a trekking exursion with her friends
  \item \textbf{P3: Luigi Verdi} is a 56 years old farmer, with a small field nearby Trento. His main products are
                             potatoes and grape, the latter one especially is cultivated in summer
                             and gathered during grape harvest period
\end{itemize}

\subsection{Competency Question}

Each Persona has to deal with different scenarios, which could be incapsuleted in the definition
of Competency Questions(CQ). Some of them could be formalized in the following table

\begin{center}
 \begin{tabular}{| l | p{10cm} |}
  \hline
  \textbf{Person} & \textbf{Question} \\
  \hline
   Mario Rossi & \textbf{Q1: }What is the trend of the temperature values in Trentino
                during the decade 2015-2025?  \\
  \hline
  Mario Rossi & \textbf{Q2: }How the metheorogical facilities are distributed in the territory? \\
  \hline
  Mario Rossi & \textbf{Q3: }Which areas in Trentino have seen the highest increase in temperature over time? \\
  \hline
  Laura Rosa & \textbf{Q4: }How was the temperature on the Trentino mountains
                during the winter of the last three years? \\
  \hline
  Laura Rosa &  \textbf{Q5: }What are the locations where rainfalls most frequently happen? \\
  \hline
  Laura Rosa &  \textbf{Q6: }What are the periods of the year where there are sunny days? \\
  \hline
  Luigi Verdi & \textbf{Q7: }Did the number of extreme climate events increased in the area around my field? \\
  \hline
  Luigi Verdi & \textbf{Q8: }What is the trend of humidity in the summer of the last three years? \\
  \hline
  Luigi Verdi & \textbf{Q9: }Did the number of rainfalls decreased over the last five years? \\
  \hline
   \end{tabular}
   \captionof{table}{Competency Questions}
\end{center}
\subsection{PurposeFormalization}

now from the scenarios, the personas, and the competency questions we extract a list of usefull concepts,
During the concept identification phase, we focused on concepts that are directly required to answer the defined
competency questions and that can be grounded in the available data sources.\\
Based on the available datasets, the core entities required to answer the defined competency questions are
\textbf{MeteorologicalFacility, MeteorologicalMeasurement and GeographicLocation}.

\begin{figure}[h!]
\centering
\includegraphics[width=\textwidth]{./PurposeFormalization.jpg}
\caption{Purpose formalization overview}
\label{fig:purpose}
\end{figure}

Structural concepts such as \textbf{Meteorological Facility} and \textbf{Meteorological Measurement} are associated with
explicit properties, as they directly correspond to entities present in the data sources.
Other concepts, including \textbf{temperature trend}, \textbf{climate variation} and \textbf{climate change indicators},
are considered derived analytical notions are inferred through queries and statistical analyses over the collected measurements.

Some concepts refer to concep already modeled in wikidata, therefore for this concept we linked the ID to the wikidata one



\subsection{ER Diagram}
At this point the next step of the formalization procedure is to produce a first bone structure
which allows us to define the involved entities more clearly. Lets first define their schemas:

\begin{center}
\begin{tabular}{| l | p{10cm} |}
\hline
\textbf{Etype} & \textbf{Properties} \\
\hline
MeteorologicalFacility & \textit{\textbf{Tavoletta n.}},\textit{Stazione, Coordinate Est/Nord, Latitudine, Longitudine, Note}\\
\hline
MeteorologicalMeasurement & \textit{\textbf{ID}}, \textit{Date, Air temperature, Min air temperature, Max air temperature,
Rain in ml, Tot rain in ml, Air relative humidity, Avg wind direction, Avg wind velocity, Atmospheric pressure, Tot solar radiation} \\
\hline
GeographicLocation & \textit{\textbf{ID}},\textit{nome, latitudine, longitudine}
\end{tabular}
\label{Entities}
\captionof{table}{Informal definition of the entities}
\end{center}

\begin{figure}[h!]
\begin{center}
\includegraphics[width=0.5\textwidth]{"./ER.jpg"}
\caption{ER diagram}
\label{ER}
\end{center}
\end{figure}
\newpage

As shown in Table~\ref{Entities}, the core entities of the Knowledge Graph represent the fundamental components
required to describe and analyze meteorological data at a regional scale.


In particular, the entity \textit{GeographicLocation} represents a generic spatial reference within the Trentino area
and is characterized by a name and geographical coordinates, which allow the association of measurements and facilities
to specific areas of interest.


The entity \textit{MeteorologicalFacility} models the physical weather stations deployed in the territory. Each facility
is uniquely identified and associated with a precise geographical location through latitude and longitude attributes.
Additional descriptive properties, such as the station name and technical identifiers, are included to support spatial
analysis and the study of the distribution of meteorological facilities across the region.


Meteorological observations are modeled through the \textit{MeteorologicalMeasurement} entity, which represents a single
measurement recorded by a meteorological facility at a given point in time. This entity includes several meteorological
variables, such as air temperature, minimum and maximum temperature, rainfall, relative humidity, wind characteristics,
atmospheric pressure, and solar radiation, together with a temporal reference.


Each meteorological measurement is associated with exactly one meteorological facility, while each facility can be
related to multiple measurements collected over time, resulting in a one-to-many relationship.


Based on the definition of these entities, the Entity–Relationship diagram shown in Figure~\ref{ER} illustrates the
structural organization of the Knowledge Graph. The separation between meteorological facilities, measurements, and
geographic locations allows the model to clearly distinguish between spatial entities, observation sources, and temporal
data. This design supports the analysis of long-term climate trends and spatial variations by enabling aggregation and
comparison of measurements across different locations and time periods.


\begin{comment}
\begin{center}
\begin{tabular}{| l | c |}
\hline
\textbf{Etype} & \textbf{Properties} \\
\hline
Location & \textit{\textbf{location\_ID}},\textit{gps\_coordinates},\textit{name} \\
\hline
Coordinates & \textit{latitude},\textit{longitude} \\
\hline
Weather & \textit{temperature},\textit{humidity},\textit{date},
\textit{wind speed},\textit{is\_sunny}  \\
\hline
MetheorogicalFacility & \textit{\textbf{facility\_ID}},\textit{name},\textit{gps\_coordinates},
\textit{inauguration\_date},\textit{still\_active} \\
\hline
ClimateEvent & \textit{\textbf{event\_ID}},\textit{location},\textit{weather\_status},\textit{event\_date} \\
\hline
\end{tabular}
\label{Entities_}
\captionof{table}{Informal definition of the entities}
\end{center}

\begin{figure}[h!]
\begin{center}
\includegraphics[width=0.5\textwidth]{"./ER.png"}
\caption{ER diagram}
\label{ER_}
\end{center}
\end{figure}
As we can see, in the table \ref{Entities} there are some entities with a more general use, like \textit{Location} and \textit{Coordinates},
which represents respectively the place where a specific event happens and the gps coordinates of a certain
position. Next we find the weather conditions associated to a location, which includes
the temperature and the humidity, the wind speed, wheter the day was sunny or not and in the end the date
timestamp of those specific metheorogical configuration. Last we have the metheorogical facility, which
include its foundation date togheter with its position, and the dangerous climate event happened at a specific
location and it is associated with the weather conditions of that day. Notice how in this latter case we also
store a further date timestamp since we want to keep track of when the catastrophe happened, which does not
necessary match the time memorized in the Weather entity (maybe because the weather conditions are sampled
every 24 hours, and the climate event happen in the meantime).
With the definition of the entities (altough informal) done it is possible now to define the ER diagram
that declares the relationship between the subjects involved, as in figure \ref{ER}. It is interesting
to notice how there is a one-to-one relationship between MetheorologicalFacility entity and Coordinates
while it could be sufficient to identify a facility through the location. The main reason why such decision
has been made is because it is possible that a metheorological station is not exactly placed inside a
specific location but it could be in the nearby.
\end{comment}


\subsection{Information Gathering}
\subsection{Evaluation - Purpose Definition}
